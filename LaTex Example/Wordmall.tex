\documentclass[12pt, a4paper]{article}
\usepackage[margin=1in]{geometry}
\usepackage{mathptmx}
\usepackage{graphicx}
\usepackage[scaled=.90]{helvet}
\usepackage{fontspec}
\setmainfont{Calibri}
\usepackage{lipsum}% just to generate filler text

\usepackage{setspace}
\doublespacing
\setlength{\parindent}{0em}
\setlength{\parskip}{1em}

\begin{document}

\begin{figure}[t]
  \includegraphics[width=2.51cm]{KTH_Logotyp_RGB_2013}
  \centering
\end{figure}

\centerline{\textit{KTH Bygg- och fastighetsekonomi}}
\centerline{\textit{AI1147 Fastighetsvärdering VT-2016}}



\vspace{1cm}
\centerline{\bfseries\large{Projekt 1 - Värdering av ett småhus}}
\vspace{1cm}


\section*{Section Headings}
We explain in this section how to obtain headings
for the various sections and subsections of our
document.

\lipsum[1-8]

\section*{Section Headings}

We explain in this section how to obtain headings
for the various sections and subsections of our
document.

\subsection*{Headings in the `article' Document Style}

In the `article' style, the document may be divided up
into sections, subsections and subsubsections, and each
can be given a title, printed in a boldface font,
simply by issuing the appropriate command.

\begin{table}[ht]
%\caption{Nonlinear Model Results}
%\centering
\begin{tabular}{l l l l}
%\hline
%\hline
\bfseries Case & \bfseries Method & \bfseries Method & \bfseries Method \\ 
%[0.5ex] % inserts table %heading
%\hline
160125	& Övning i datasal	&	837	& 970 \\
2 & 47 & 877 & 230 \\
3 & 31 & 25  & 415 \\
4 & 35 & 144 & 2356 \\
5 & 45 & 300 & 556 \\ [1ex]
%\hline
\end{tabular}
\label{table:nonlin}
\end{table}


\subsection*{Syfte}

Syftet med detta projekt är att ni ska få övning i att praktiskt tillämpa era kunskaper kring värdering av småhusfastigheter. 

\subsection*{Mål}
Efter avslutat projektarbete ska ni kunna:
•	Bedöma marknadsvärdet för en småhusfastighet
•	Redogöra för hur man principiellt kan gå till väga för att ta fram värdeutlåtanden för småhusfastigheter

\subsection*{Instruktioner}
Genom att använda er av areametoden och köpeskillingskoefficientmetoden som finns beskrivna i kapitel 19 i kurslitteraturen ska ni bedöma marknadsvärdet för en given fastighet, med försäljningstidpunkt den 1 februari 2016.
Ni kommer att arbeta i grupper om 2-3 studenter, där varje grupp värderar ett av två valbara objekt. Grundläggande information om fastigheten som ni ska värdera finns på Bilda under mappen DOKUMENT/PROJEKT 1 - SMÅHUS.
Information om marknadsförutsättningar kan ni hitta hos respektive kommun, lokala mäklarfirmor m.m.
Er främsta källa för information om överlåtelser i området kring fastigheten kommer att vara NAI-Svefas Ortsprissystem. På NAI-Svefas hemsida använder ni applikationen ”Real Estate” (användarnamn: ext_kth02, lösenord: Våren2015). All data om jämförelseobjekt som hämtas ur NAI-Svefas Ortprissystem skall sedan exporteras till MS-Excel där ni kommer att utföra alla nödvändiga beräkningar. 

\end{document}

